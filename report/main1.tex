\documentclass[12pt,a4paper,titlepage]{report}
\usepackage{graphicx}
\usepackage[left=2.8cm, right=2.2cm, top=3cm, bottom=2.5cm]{geometry}
\usepackage{latexsym}		% to get LASY symbols
\usepackage{epsfig}			% to insert PostScript figures
\usepackage{eufrak}
\usepackage{type1cm}
\usepackage{newsec}
\usepackage{titlesec}
\usepackage{longtable}
\usepackage{url}
\usepackage[export]{adjustbox}
\usepackage{listings}
\usepackage{xcolor}
\usepackage{pdfpages}
\usepackage{caption}
\usepackage{listings}
\usepackage{float}
\usepackage{acro}

\definecolor{customgreen}{rgb}{0,0.6,0}
\definecolor{customgray}{rgb}{0.5,0.5,0.5}
\definecolor{custommauve}{rgb}{0.6,0,0.8}
\lstdefinelanguage{HTML}{
	sensitive=true,
	keywords={},
	otherkeywords={<, >, /},
	morecomment=[s]{<!--}{-->},
	morestring=[b]"
}


\lstset{ 
	basicstyle=\small,        % the size of the fonts that are used for the code
	breaklines=true,                 % sets automatic line breaking
	commentstyle=\color{customgreen},    % comment style
	firstnumber=1,                % start line enumeration with line 1000
	frame=single,	                   % adds a frame around the code
	keepspaces=true,                 % keeps spaces in text, useful for keeping indentation of code (possibly needs columns=flexible)
	keywordstyle=\color{blue},       % keyword style
	numbers=left,                    % where to put the line-numbers; possible values are (none, left, right)
	numbersep=10pt,                   % how far the line-numbers are from the code
	numberstyle=\tiny\color{customgray}, % the style that is used for the line-numbers
	rulecolor=\color{black},         % if not set, the frame-color may be changed on line-breaks within not-black text (e.g. comments (green here))
	showspaces=false,                % show spaces everywhere adding particular underscores; it overrides 'showstringspaces'
	showstringspaces=false,          % underline spaces within strings only
	showtabs=false,                  % show tabs within strings adding particular underscores
	stepnumber=1,                    % the step between two line-numbers. If it's 1, each line will be numbered
	stringstyle=\color{custommauve},     % string literal style
	tabsize=2,	                   % sets default tabsize to 2 spaces
	title=\lstname                   % show the filename of files included with \lstinputlisting; also try caption instead of title
}


\titleformat{\chapter}[display]
{\normalfont\Large\bfseries\centering}{\chaptertitlename\
	\thechapter}{25pt}{\Large}
\titleformat{\section}{\normalfont\bfseries}{\noindent\thesection}{20pt}{}
\titleformat{\subsection}{\normalfont\small\bfseries}{\thesubsection}{15pt}{\small}
\titlespacing*{\chapter}{0pt}{0pt}{40pt}


\begin{document}
	\titlepage
	\thispagestyle{empty}
	\begin{center}
		\includegraphics[scale=0.3]{logo1.png}\\[0.5cm]
		\large \textit{Mini Project Report On}\\[0.6cm]
		\Large \textbf{Grade AI (An AI Tool that evaluates Answer Sheets) }\\[0.6cm]
		\textit{Submitted in partial fulfillment of the
			requirements for the award of the degree of}\\[0.6cm]
		{\huge {$\mathfrak {Bachelor\; of\; Technology}$}}\\[.2cm]
		%{\Large {$\mathfrak {In}$}}\\[.5cm]
		%{\Large {$\mathfrak {Computer\; Science\; \&\; Engineering}$}}\\[2cm]
		\textit{in}\\[.2cm]
		{\Large \bf \itshape{{Computer\; Science\; \&\; Engineering}}}\\[0.4cm]
		\large \bfseries{By}\\[.4cm]
		\large \bfseries{ Abhinav Sobi (U2103008) }\\[0.2cm]
		\large \bfseries{ Alan Joseph (U2103021) }\\[0.2cm]
		\large \bfseries{ Basil Eldho Joseph (U2103057) }\\[0.2cm]
		\large \bfseries{ Daniel Robin (U2103072) }\\[0.6cm]
		\large \bfseries{Under the guidance of}\\[0.75cm]
		\large \bfseries{Dr. Dhanya P.M.}\\[0.75cm]
%		\includegraphics[width=8.0cm]{logo (1).jpg}\\[0.5cm]
		\large \textbf{Department of Computer\; Science\; \&\; Engineering}\\
		\large \textbf{Rajagiri School of Engineering \&\ Technology (Autonomous)}\\
		\small \bfseries{(Affiliated to APJ Abdul Kalam Technological University)}\\
		\large \textbf{Rajagiri Valley, Kakkanad, Kochi, 682039}\\
		\large \bfseries{May 2024}
	\end{center}
	
	\newpage
	\thispagestyle{empty}
	\begin{center}
	%	\textbf {DEPARTMENT OF COMPUTER SCIENCE \&\ ENGINEERING}\\
	%	\small \textbf{RAJAGIRI SCHOOL OF ENGINEERING \&\ TECHNOLOGY (AUTONOMOUS)}\\
	
	%	\small \textbf{RAJAGIRI VALLEY, KAKKANAD, KOCHI, 682039}\\
     %   \small \bfseries{(Affiliated to APJ Abdul Kalam Technological University)}\\[0.5cm]
		%\begin{figure}[htbp]
		%	\centering
			%\includegraphics[scale=0.40]{logo (1).jpg}
         %   \includegraphics[width=8.0cm]{logo (1).jpg}\\[0.5cm]
		%\end{figure}
		\large \bfseries{\huge{CERTIFICATE}}\\[5cm]
	\end{center}
	
	\renewcommand{\baselinestretch}{1.2}\normalsize
	
	\emph{This is to certify that the mini project report entitled \textbf{”Grade AI”} is a bonafide record of the work done by \textbf{Abhinav Sobi (U2103008)}, \textbf{Alan Joseph (U2103021)}, \textbf{Basil Eldho Joseph (U2103057)}, \textbf{Daniel Robin (U2103072)}, submitted to the APJ Abdul Kalam Technological University in 
		partial fulfillment of the requirements for the award of the degree of Bachelor of Technology (B. Tech.) in Computer Science and Engineering during the academic year 2023-2024.}\\[2.5cm]
	
\begin{flushleft}
	
	
	\begin{longtable}{p{10.3cm} p{8cm} p{5.25cm}}
		\small\textbf{Dr. Dhanya P.M.}    &\small\textbf{Mr. Harikrishnan M} \\
		{Project Guide}& {Project Coordinator}\\
        {Professor}& {Asst. Professor}\\
		{Dept. of CSE}&  {Dept. of CSE}\\
		{RSET} ~&{RSET} \\
		
	\end{longtable}
\end{flushleft}
	
\begin{center}
	\small\textbf{Dr. Preetha K. G.}\\
	{Head of the Department}\\
	{Professor}\\
	{Dept. of CSE}\\
	{RSET}
\end{center}	

	
	

    
	
	\renewcommand{\baselinestretch}{1.5}\normalsize
	\newpage
	%\renewcommand\abstractname{ACKNOWLEDGEMENTS}
    \chapter*{ACKNOWLEDGEMENTS}
	%\begin{abstract}
		\pagenumbering{roman}
		\setcounter{page}{1}
		\addcontentsline{toc}{chapter}{Acknowledgements}
		\vspace{1.5cm}
		%\begin{spacing}{}
		\paragraph\ I wish to express my sincere gratitude towards Dr P. S. Sreejith, Principal of RSET, and Dr. Preetha K.G., Head of the Department of Computer Science and Engineering for providing me with the opportunity to undertake my mini project, "Grade AI".
		\paragraph\ I am highly indebted to my project coordinators, \textbf{Mr. Harikrishnan M}, Assistant Professor, Department of Computer Science and Engineering and \textbf{Ms. Sherine Sebastian}, Assistant Professor, Department of Computer Science and Engineering for their valuable support.
		
		\paragraph\ It is indeed my pleasure and a moment of satisfaction for me to express my sincere
		gratitude to my project guide \textbf{Dr. Dhanya P.M.} for her patience and all the priceless advice and  wisdom she has shared with me.
		\paragraph\ Last but not the least, I would like to express my sincere gratitude towards all other teachers and friends for their continuous support and constructive ideas.
		
		%\end{spacing}
		\begin{flushright}
			\textbf{Abhinav Sobi}\\
			\textbf{Alan Joseph}\\
			\textbf{Basil Eldho Joseph}\\
			\textbf{Daniel Robin}
		\end{flushright}
		
		
	%\end{abstract}
   
	
	\newpage
 
	\renewcommand{\baselinestretch}{1.5}\normalsize
	%\thispagestyle{empty}
	%\renewcommand\abstractname{ABSTRACT}
    \chapter*{Abstract}
	%\begin{abstract}
	%	\pagenumbering{roman}
	%	\setcounter{page}{3}
		\addcontentsline{toc}{chapter}{Abstract}
		\vspace{1.5cm}
		\paragraph\ \ \ \ 
            Grade AI is a web-based AI Answer Paper Evaluator which aims to revolutionize the education sector by automating the exam evaluation process. Institutions can effortlessly submit their exam papers, get them evaluated and view statistics and feedback within a single, user-friendly interface accessible from any device with internet connectivity. 
            \paragraph\ The web application will allow institutions to create an account to view their exams history and results. New exam submissions can be created, following which the hand-written answer sheets are scanned and uploaded as PDF or image files. The answer key, based on which the answer sheets are to be evaluated, is also uploaded. The uploaded papers will be securely processed and stored on the server for evaluation.
            \paragraph\ The system will speed up the evaluation process by a huge factor when compared to the traditional evaluation process conducted by humans. It also helps to reduce biases and discrepancies which may occur due to multiple human evaluators. The answer papers will be evaluated based on the relevance, coherence, quality and semantic similarity of answers with respect to the answer key. Based on this, the system will automatically assign scores to the answer papers.
            \paragraph\ Exam results can be shared by the examination cell of the institution to the students, teachers and the public as required. The system also allows students to log-in and view their results, grades and rank.
            Overall, the AI Answer Paper Evaluator aims to redefine the evaluation process, reduce the workload on teachers, and provide more consistent and objective assessments of student performance, ultimately enhancing the quality of education.
		
		%\end{spacing}
	%\end{abstract}
	
	
	\newpage
	\normalsize{}
	
	%\pagenumbering{roman}
	%\setcounter{page}{4}
	%\begin{spacing}{}
	\tableofcontents
	%\end{spacing}
	% \thispagestyle{empty}
	\newpage
	

	\listoffigures
	\addcontentsline{toc}{chapter}{List of Figures}
	% \thispagestyle{empty}
	\newpage
	

   % \pagenumbering{roman}
	%\setcounter{page}{9}
	\listoftables
	\addcontentsline{toc}{chapter}{List of Tables}
%	\addcontentsline{toc}{chapter}{List of Abbreviations}
%	\listof{Abbreviations}
	% \thispagestyle{empty}
    \newpage
    
    \chapter*{List of Abbreviations}
    \addcontentsline{toc}{chapter}{List of Abbreviations}
    
    Acronym - Expansion \\
    
    \newpage
    
	\cleardoublepage
	
	\pagenumbering{arabic}
	\setcounter{page}{1}
	%\begin{spacing}{}
	\chapter{Introduction}
	
	\section{Background}

    Handwritten Exam paper evaluation is a hassle which troubles both educators and students alike. For the teachers or professors, exam evaluation is a tedious task which takes up a significant amount of their time and energy. The human biases arising from a diverse group of evaluators affects the quality of the evaluation. Students have to wait months before the exam results get published, this affects their future studies. Viewing the evaluation and applying for re-evaluation is also a hassle which takes up even more time and resources.\\
    Hence the Education sector is in dire need of a swift, streamlined and accurate Exam Evaluation and Result Management process. 
    
	
	\section{Problem Definition}

    To develop a platform for automatic grading of handwritten exam answer papers based on their semantic similarity, keyword matching, answer length and more.\\
    The platform should also provide a seamless and user friendly result viewing and management experience.
	
	
	\section{Scope and Motivation}
	
	\paragraph\ GradeAI has the scope to revolutionize the Educational sector. It aims to accelerates the text based exam evaluation.\\
    The project is not suitable for similarity analysis of diagrams or graphs.\\
    GradeAI evaluates any answer sheets consisting of English text only.\\
    
    \paragraph\ Motivation for the project are the countless unpleasant experiences associated with Exam evaluation which has plagued both students and educators throughout the years.\\
    Teachers have to spent hours, even after midnight, to complete the evaluation.\\
    Delay in result publishing has resulted in rejection of admissions for higher studies.\\
    Students get marks which are less than they deserve due to human biases or errors.\\
    Re-evaluation is a tedious and expensive process which takes up more time.
	
	You may insert tables into your document using the given code:
	\begin{table}[!h]
		\begin{center}
			\begin{tabular}{|c|c|c|}
				\hline
				\textbf{Title 1} & \textbf{Title 2} & \textbf{Title 3} \\
				\hline
				1 & Content 1 & Content 2 \\
				\cline{1-2}
				2 & Content 3 & Content 4\\
				\hline
			\end{tabular}
			\caption{Insert tables here}
		\end{center}
	\end{table}

	
	\section{Objectives}
	
	\begin{itemize}
        \item[1.] GradeAI should be an easily accessible and user friendly platform for both educators and students.
        \item[2.] Educator should be able to scan and upload answer keys and answer papers.
        \item[3.] GradeAI should evaluate the answers of the students with respect to the answer key provided based on semantic similarity.
        \item[4.] An exam administrator, who represents the educational institute, must be able to view and manage the data.
        \item[5] Students must be able to view their results, grade and rank for the exams they attended. 
	\end{itemize}
	
	\section{Challenges}
	
	Optical Character Recognition (OCR) technology is not advanced enough to perfectly extract text from poor hand-writings. OCR is also very resource intensive. 
	
	
	\section{Assumptions}
	
	The availability and stability of the Google Gemini API is a critical factor in the project. The project assumes the API remains relatively stable in terms of functionality and response format.\\
    The project assumes a certain level of OCR accuracy. Poor OCR results will render subsequent text comparison less meaningful, potentially requiring either manual intervention or a very flexible matching process tolerant of errors introduced by the OCR.
	
	
	\section{Societal / Industrial Relevance}

    Grade AI is relevant in the Education sector, from elementary schools to professional colleges.
    Any educational institution who conducts handwritten exams can use the services of the project.
	This section describes where the project can be applied, either for the society or the industry. Write the relevance applicable for the work.
	
	\section{Organization of the Report}
	
	This section should outline a roadmap of the contents in the report.
	
	
	
	
	
	
	
	\chapter{Software Requirements Specification}
	Insert your SRS document here.
	
	\section{Introduction}
	\paragraph\ 
	
	\begin{figure}[htbp]
		\centering
		%\includegraphics[scale=0.40]{logo (1).jpg}
		\includegraphics[width=8.0cm]{logo (1).jpg}\\[0.5cm]
		\caption{Insert your images here, and provide necessary captions}
	\end{figure}
	
	\section{Overall Description}
	\paragraph\ 
	
	\section{External Interface Requirements}
	\paragraph\ 
	
	You can insert equations into your file using the below code:
	

	\begin{eqnarray}
		a & = & b + c \\
		& = & y - z
	\end{eqnarray}
	
	\section{System Features}
	\paragraph\ 
	
	\section{Other Nonfunctional Requirements}
	\paragraph\ 
	

	
	
	
	\chapter{System Architecture and Design}
	\section{System Overview }
	\paragraph\ 
	This section gives an overview of the project. Detailed architecture diagram is expected in
	this section. The entire process has to be outlined in detail. You can write upto 2 pages.
	
	\section{Architectural Design }
	\paragraph\ 
	This section can use the design tools for a software product development like use case
	diagram, sequence diagram, ER diagram etc. No textual description is required. The titles for
	these diagrams should be chosen carefully so that they are self explanatory.
	
	\section{Dataset identified}
	\paragraph\ 
	This section describes the data source used in the project. Brief its properties and refer it to
	the appropriate location. Sample subsets of the dataset can be highlighted.
	
	\section{Proposed Methodology/Algorithms}
	\paragraph\ 
	This section describes in detail the methodologies or algorithms associated with your work.
	Algorithms should be written in appropriate format.
	
	
	\section{User Interface Design}
	\paragraph\
	The user interface design (wireframe designs) can be highlighted in this section. The figures
	titles should be in a chronological order and self explanatory.
	
	
	
	\section{Database Design}
	\paragraph\
	The detailed database design and its schema is expected in this section. The database used
	in the work can be mentioned here. The reason for choosing the database can be
	substantiated in this section.
	
	
	
	\section{Description of Implementation Strategies}
	\paragraph\
	"This section details the important implementation strategies used in your project. For
	example, (1) if you are doing a audio/ video capture, what is your python library and
	associated methods for doing that (2) if you are designing a CNN, how did you do that in the
	language which you are using for implementation (3) what are the different methods in the
	language you will use to evaluate your work. Small snippets of code can be supplemented to
	support your strategy."
    Libraries Used:
    \begin{itemize}
        \item[1.] google.generativeai:
       \paragraph\
       This library provides tools for generative AI tasks, enabling users to create or train models for tasks such as image generation, text generation, and more.\\
       - **Methods:** Utilize pre-trained models or train custom models using techniques like GANs (Generative Adversarial Networks) or VAEs (Variational Autoencoders) for various generative tasks.
    
    \item[2.] fastapi:
       \paragraph\
       FastAPI is a modern, fast (high-performance) web framework for building APIs with Python 3.7+.\\
       - **Methods:** Define API endpoints using FastAPI decorators, handle request and response validation using Pydantic models, and utilize asynchronous programming for high concurrency.
    
    \item[3.] selenium:
       \paragraph\
       Selenium is a web testing tool that can automate browser actions and interactions with web pages.\\
       - **Methods:** Automate interactions with web pages, such as filling forms, clicking buttons, and scraping data from websites.
    
    \item[4.] pydantic:
       \paragraph\
       Pydantic is a data validation and settings management library in Python.\\
       - **Methods:** Define Pydantic models to represent data structures, validate incoming data, and serialize data to and from JSON.
    
    \item[5.] sqlalchemy:
       \paragraph\
       SQLAlchemy is a SQL toolkit and Object-Relational Mapping (ORM) library for Python.\\
       - **Methods:** Define database models using SQLAlchemy ORM, perform CRUD operations (Create, Read, Update, Delete), and execute raw SQL queries for database manipulation.
    
    \item[6.] alembic:
       \paragraph\
       Alembic is a database migration tool for SQLAlchemy, providing a way to manage changes to a relational database schema over time.\\
       - **Methods:** Create and manage database migrations to version control database schema changes, allowing for easy upgrades and downgrades of database schemas.
	\end{itemize}
 
	\section{Module Division}
    \paragraph\
    "This section describes the different modules involved in this project and a small description
	of the same is expected. This section ends with the information of which module is assigned
	to each project member."
	\begin{itemize}
        \item[1.] User Interface Module:\\
         This module involves creation and implementation of the user interface for Grade AI. It handles the design and functionality of features like user authentication, answer sheet uploading, and display of results.
        \item[2.] OCR Integration Module:\\
        This module handles the integration of Optical Character Recognition (OCR), Handwritten Text Recognition (HTR) technology into the system. It processes uploaded answer sheets to extract text data to be used in the evaluation process.
        \item[3.] Answer Correction Module:\\
        This module implements the algorithm for correcting answers based on extracted text data and a predefined answer key. It evaluates the similarity and correctness of answers.
        \item[4.] Database Management Module:\\
        This module manages the storage and retrieval of answer sheets, corrected results, and associated metadata. It ensures data integrity and security.
	\end{itemize}
    1. User Interface Module - \\
    2. OCR Integration Module - \\
    3. Answer Correction Module - \\
    4. Database Management Module - 
    
	\section{Work Schedule - Gantt Chart}
	\paragraph\ 
	Outline the work schedule in the appropriate chart format.
	

	
%	\chapter{Results and Discussions}
	
	
%	\section{Overview}
%	\paragraph\
%	This section describes the overall results achieved in terms of the end results, quantitative	results and further analysis. One paragraph of textual description is expected.
	
	
%	\section{Testing}
%	\paragraph\
	
%	For a webapp/database project, screenshots of results in chronological order can be added	in this section. Other types of projects also can have this section with less length.
	
	
%	\section{Quantitative Results}
%	\paragraph\
%	The quantitative results of the project (eg- numerical values like accuracy, precision, rmse,	confusion matrix etc ) can be supplemented in this section. Important note is textual description of all results is mandatory. Give appropriate titles for the tabular results.
	
	
%	\section{Graphical Analysis}
%	\paragraph\
%	The graphical analysis of the project can be given in this section. Choose your graph	representation in accordance with your project. Important note is textual description of allvresults is mandatory. Give appropriate titles for the graphical results. In this section, comparison of your results with other paper titles mentioned in Chapter 2 are also encouraged.
	
	
%	\section{Discussion}
%	\paragraph\
%	This section describes a summary of the results. You are also welcome to substantiate the reason behind your results or also the deviation of the results.
	
%	\chapter{Conclusion}
	
	
%	\section{Conclusion}
%	\paragraph\
%	This section describes the conclusion of the project in one page. Write one or two paragraphs.
	
	
%	\section{Future Scope}
%	\paragraph\
%	In this section outline the future scope/extensions possible in the project in four or five	sentences.
	
%	\newpage
%	\normalsize{}

%	\chapter*{List of Publications}
%	\addcontentsline{toc}{chapter}{Publications}
%	\begin{enumerate}
%		\item Publication 1
%		\item Publication 2
%	\end{enumerate}
		
		
%	\clearpage
%	\renewcommand{\bibname}{References} 
%	\addcontentsline{toc}{chapter}{References}
	
	\begin{thebibliography}{99}
	\bibitem a H. Garg and M. Dave, "Securing IoT Devices and SecurelyConnecting the Dots Using REST API and Middleware," 2019 4th International Conference on Internet of Things: Smart Innovation and Usages (IoT-SIU), Ghaziabad, India, 2019, pp. 1-6, doi: 10.1109/IoT-SIU.2019.8777334.
	\bibitem b M. Ebrahimi, Y. Chai, H. H. Zhang and H. Chen, "Heterogeneous Domain Adaptation With Adversarial Neural Representation Learning: Experiments on E-Commerce and Cybersecurity," in IEEE Transactions on Pattern Analysis and Machine Intelligence, vol. 45, no. 2, pp. 1862-1875, 1 Feb. 2023, doi: 10.1109/TPAMI.2022.3163338.
	\bibitem c Reference 3
	\bibitem d Reference 4
	\bibitem e Reference 5
	\bibitem f Reference 6
	\bibitem g Reference 7
	\bibitem h Reference 8
	\bibitem i Reference 9
	\bibitem j Reference 10
	
	\end{thebibliography}
	
	
%	\clearpage
%	\addcontentsline{toc}{chapter}{Appendix A: Presentation}
%	\chapter*{}
%	\paragraph\ 
%	\vspace{75mm}
%	\begin{center}
%		\textbf{\huge{Appendix A: }}
%		\textbf{\huge{Presentation}}
%	\end{center}
%	\includepdf[pages=-]{1.pdf}
	
%	\clearpage
	%\clearpage
	
%	\addcontentsline{toc}{chapter}{Appendix B: Vision, Mission, Programme Outcomes and Course Outcomes}
%	\chapter*{}
%	\paragraph\ 
%	\vspace{75mm}
%	\begin{center}
%		\textbf{\huge{Appendix B: }}
%		\textbf{\huge{Vision, Mission, Programme Outcomes and Course Outcomes}}
%	\end{center}
%	\includepdf[pages=-]{2.pdf}
%	\clearpage
%	\clearpage
%	 \newpage
	%\cleardoublepage
%	\thispagestyle{empty}
%	\begin{center}
		%	\textbf {DEPARTMENT OF COMPUTER SCIENCE \&\ ENGINEERING}\\
		%	\small \textbf{RAJAGIRI SCHOOL OF ENGINEERING \&\ TECHNOLOGY (AUTONOMOUS)}\\
		%	\small \textbf{RAJAGIRI VALLEY, KAKKANAD, KOCHI, 682039}\\
		%	\small \bfseries{(Affiliated to APJ Abdul Kalam Technological University)}\\[0.5cm]
		%	\begin{figure}[htbp]
			%		\centering
			%\includegraphics[scale=0.40]{logo (1).jpg}
			%		\includegraphics[width=8.0cm]{logo (1).jpg}\\[0.5cm]
			%	\end{figure}
%		\large \bfseries{Vision, Mission, Programme Outcomes and Course Outcomes}
%	\end{center}
	%	\pagenumbering{roman}
	%	\setcounter{page}{2}
%	\addcontentsline{toc}{chapter}{Vision, Mission, POs, PSOs and COs}
%	\vspace{1.5cm}
%	\renewcommand{\baselinestretch}{1.2}\normalsize
	
%	\textbf{Institute Vision} \\
%	To evolve into a premier technological institution, moulding eminent professionals with creative minds, innovative ideas and sound practical skill, and to shape a future where technology works for the enrichment of mankind. \\ \\
	
%	\textbf{Institute Mission} \\
%	To impart state-of-the-art knowledge to individuals in various technological disciplines and to inculcate in them a high degree of social consciousness and human values, thereby enabling them to face the challenges of life with courage and conviction. \\ \\
	
%	\textbf{Department Vision} \\
%	To become a centre of excellence in Computer Science and Engineering, moulding professionals catering to the research and professional needs of national and international organizations. \\ \\
	
%	\textbf{Department Mission} \\
%	To inspire and nurture students, with up-to-date knowledge in Computer Science and Engineering, ethics, team spirit, leadership abilities, innovation and creativity to come out with solutions meeting societal needs. \\ \\
	
%	\textbf{Programme Outcomes (PO)} \\
%	Engineering Graduates will be able to: \\ \\
%	\textbf{1. 	Engineering Knowledge}: Apply the knowledge of mathematics, science, engineering fundamentals, and an engineering specialization to the solution of complex engineering problems. \\ \\
%	\textbf{2.	Problem analysis}: Identify, formulate, review research literature, and analyze complex engineering problems reaching substantiated conclusions using first principles of mathematics, natural sciences, and engineering sciences. \\ \\
%	\textbf{3.	Design/development of solutions}: Design solutions for complex engineering problems and design system components or processes that meet the specified needs with appropriate consideration for the public health and safety, and the cultural, societal, and environmental considerations. \\ \\
%	\textbf{4. Conduct investigations of complex problems}: Use research-based knowledge including design of experiments, analysis and interpretation of data, and synthesis of the information to provide valid conclusions. \\ \\
%	\textbf{5.	Modern Tool Usage}: Create, select, and apply appropriate techniques, resources, and modern engineering and IT tools including prediction and modeling to complex engineering activities with an understanding of the limitations. \\ \\
%	\textbf{6.	The engineer and society}: Apply reasoning informed by the contextual knowledge to assess societal, health, safety, legal, and cultural issues and the consequent responsibilities relevant to the professional engineering practice. \\ \\
%	\textbf{7.	Environment and sustainability}: Understand the impact of the professional engineering solutions in societal and environmental contexts, and demonstrate the knowledge of, and need for sustainable development. \\ \\
%	\textbf{8.	Ethics}: Apply ethical principles and commit to professional ethics and responsibilities and norms of the engineering practice. \\ \\
%	\textbf{9. Individual and Team work}: Function effectively as an individual, and as a member or leader in teams, and in multidisciplinary settings. \\ \\
%	\textbf{10.	Communication}: Communicate effectively with the engineering community and with society at large. Be able to comprehend and write effective reports documentation. Make effective presentations, and give and receive clear instructions. \\ \\
%	\textbf{11.	Project management and finance}: Demonstrate knowledge and understanding of engineering and management principles and apply these to one's own work, as a member and leader in a team. Manage projects in multidisciplinary environments. \\ \\
%	\textbf{12.	Life-long learning}: Recognize the need for, and have the preparation and ability to engage in independent and lifelong learning in the broadest context of technological change. \\ \\
	
%	\textbf{Programme Specific Outcomes (PSO)} \\
%	A graduate of the Computer Science and Engineering Program will demonstrate: \\ \\
%	\textbf{PSO1: Computer Science Specific Skills} \\
%	The ability to identify, analyze and design solutions for complex engineering problems in multidisciplinary areas by understanding the core principles and concepts of computer science and thereby engage in national grand challenges. \\ \\
%	\textbf{PSO2: Programming and Software Development Skills} \\
%	The ability to acquire programming efficiency by designing algorithms and applying standard practices in software project development to deliver quality software products meeting the demands of the industry. \\ \\
%	\textbf{PSO3: Professional Skills} \\
%	The ability to apply the fundamentals of computer science in competitive research and to develop innovative products to meet the societal needs thereby evolving as an eminent researcher and entrepreneur. \\ \\
	
%	\textbf{Course Outcomes} \\
	

	
%	\clearpage
	
%	\addcontentsline{toc}{chapter}{Appendix C: CO-PO-PSO Mapping}
%	\chapter*{}
%	\paragraph\ 
%	\vspace{75mm}
%	\begin{center}
%		\textbf{\huge{Appendix C: }}
%		\textbf{\huge{CO-PO-PSO Mapping}}
%	\end{center}
%	\includepdf[pages=-]{3.pdf}
%	\clearpage
%	\clearpage
	
\end{document}